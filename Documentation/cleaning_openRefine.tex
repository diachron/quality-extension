\section{OpenRefine extension}


general description of service, its components and their functionalities


\section{The OpenRefine Extension for Cleaning RDF Data}
\label{sec:cleaning}

\subsection{Cleaning Workflow}

\subsubsection{Extension Points}

Picture of extension points.

The described workflow is shown in the picture (screenshort of entry point).
Required file formats. (reference to jena library)


Generel description of two moduls. 
Short description of each module - Open refine extension - interaktive cleaning required user involvment.

Cleaning service - automatic cleaning.


D3.2 presents a first version of cleaning workflow. 

\subsubsection{Cleaning with OpenRefine}

1. Display RDF data set in a table form with "`subject, predicate, object"' columns -  in order to provide understable data visualization.
2. Identification of quality Problems (war described in D3.2 The changed UI is shown in figure...)
3. Generation of cleaning suggestion (for some metrics, e.g Undefined classes or undefined properties were implemented extended suggestions).
4. Application of cleaning rules
5. Export of cleaned data (screenshort and available filr formats), Statistic report



Cleaning workflow as a diagramm
Description of each step separately:
\begin{enumerate}
	
\item Load Project
\item Select metrics
\item Clean data
 \item Export cleaned data
\end{enumerate}
\subsection{Quality Problems}
Ruslan
\subsection{Cleaning Suggestions}


\subsection{Technical Documentation}
\subsubsection{Architecture design}
UML diagramm

\subsubsection{Prerequisites}
updates ontologies, 
new libraries




\subsubsection{How to install}

Install OpenRefine
Check out extension

Ruslan
\subsubsection{How to run}

Ruslan
\








%%% Local Variables: 
%%% mode: latex
%%% TeX-master: "D3.2"
%%% End: 
