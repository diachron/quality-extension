\section{Introduction}
\label{sec:intro}

This is a technical deliverable, in which we provide technical documentation regarding the implementation of the first version of the DIACHRON services of WP3.
In addition, we provide the final version of the specifications of these services, with emphasis on any differences with respect to what was reported in Deliverable D3.1~\cite{d3.1}.
The services have already been implemented and we have uploaded the source code in the project's github account for internal use by the partners (\url{https://github.com/diachron/detection_repair}). 
Moreover, we have publicly deployed the services under URL \url{http://139.91.183.93:8181} (Apache Tomcat 7.0.54) as a stand-alone module. However, they will also be formally deployed as part of the integrated DIACHRON environment for Deliverable D6.2 in M20.
In the following subsections, we describe the services that will be considered in this deliverable (and have been promised in the context of WP3), and a reference to their detailed description in this report. The appendix contains the up-to-date listings of the changes considered in DIACHRON.

\subsection{Change Detection Service}

The \textit{change detection service} is responsible for identifying the changes that occurred between any two given versions (and led from one version to the other). 
Said detection occurs \emph{after} the change has happened, i.e., the system assumes no knowledge on the change process itself, and the only input for performing the detection is the content of the two versions to be compared.
The implementation of this service abides by the definitions of the changes listed in the appendices, and its output should store the detected changes in the ontology of changes (cf.~\cite{d3.1} and Subsection~\ref{subsec:chdet_description}).

The change detection service is described in Section~\ref{sec:detection}.
Emphasis is given to the modifications made in the original design (compared to D3.1~\cite{d3.1}), which was based on the pilots' feedback and internal discussions; these changes are described in 
Subsection~\ref{subsec:chdet_description}.

\subsection{Change Monitoring and Propagation Service}

The \emph{change monitoring and propagation} service provides publish-subscribe functionality for creating change monitoring tasks and notifications. A monitoring task is a process configured by the user where a dataset is periodically inspected for changes. Change detection is performed by employing the Change Detection module functionality. The user may configure the task by declaring rules on the types of changes and the dataset resources she wishes to monitor. Other users subscribe to one or more monitoring tasks, so that notifications for detected changes are propagated to them.

The design of this service and its API specification have been presented in documents D3.1 and D6.1 and, as described in these documents, the \emph{change monitoring and propagation} service is heavily dependent upon the archive and the change detection services. Therefore, the implementation process and especially the testing and the integration phases of the service need to invoke the stable versions of these relying services. The first prototype versions of the archive and change detection services are scheduled to be delivered in M16 of the project. As a result the delivery of the monitoring and propagation service will follow this date and thus be included in the second prototype of the WP3 services scheduled in M28.

\subsection{Repairing Service}

The \emph{repairing service} deals with the problem of identifying and resolving invalidities in datasets. 
An \emph{invalidity} is defined as a violation of a certain constraint associated with the underlying data, such as the requirement for two concepts to be disjoint.
The repairing service of DIACHRON considers constraints of logical nature, in particular constraints that can be expressed in the language \dllite{A}~\cite{dl-litea}. It provides an efficient methodology for identifying invalidities (taking into account \dllite{A} reasoning), as well as for resolving them in a manner that has the least impact (in terms of lost knowledge) to the dataset. 

The details on the theoretical background upon which this service is based have been described in~\cite{d3.1}. This deliverable contains a listing of the related algorithms, as well as some technical documentation on the implementation of the service  (see Section~\ref{sec:repairing}).

\subsection{Cleaning Service}

In contrast to the repairing service, which deals with inconsistencies of a logical nature, the cleaning service aims to address a different aspect of data quality: it verifies whether 
small pieces of information are logically consistent in themselves, whether they are well-formed, and whether their representation follows good practices of data engineering and data publishing.
Since the correction of such problems requires user involvement and often detailed knowledge about the data, the cleaning process cannot be performed fully automatically.
The cleaning \emph{service} therefore focuses on reporting cleaning suggestions and offers automated cleaning by deleting any data with quality problems.
For those more advanced cleaning tasks that inherently require user interaction, we offer a cleaning \emph{application}, which shares code with the cleaning service.
The main purpose of the cleaning service and the cleaning application is to support the user in the following data cleaning issues: i) automatic identification of inconsistencies/anomalies in the data and
%\todo{CL@NF: This was ``iii)'' but there was no ``ii)''.  Could you please check whether we have no lost anything here?}  
ii) generation of suggestions addressing the identified problems.
The general design of the cleaning service has appeared in D3.1~\cite{d3.1}; this deliverable (in particular, Section~\ref{sec:cleaning}) provides its technical documentation.

%%% Local Variables: 
%%% mode: latex
%%% TeX-master: "D3.2"
%%% End: 
